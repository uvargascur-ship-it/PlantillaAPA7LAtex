% =========================================================================
%      PLANTILLA MAESTRA DE TESIS - UNMSM (FACULTAD DE CIENCIAS ECONÓMICAS)
%                             Normas APA 7ma Edición
% =========================================================================
% Autor: Ulises Vargas
% Versión: 1.0 (2026)
% =========================================================================

\documentclass[12pt, a4paper]{article}

% -------------------------------------------------------------------------
% 1. CODIFICACIÓN E IDIOMA
% -------------------------------------------------------------------------
\usepackage[utf8]{inputenc} % Permite escribir tildes y ñ directamente
\usepackage[T1]{fontenc}    % Asegura la correcta salida de caracteres en el PDF
\usepackage[spanish, es-tabla, es-nodecimaldot]{babel}
% es-tabla: Cambia "Cuadro" por "Tabla" (Requisito APA).
% es-nodecimaldot: Usa punto (.) para decimales en lugar de coma.

% -------------------------------------------------------------------------
% 2. FORMATO DE PÁGINA Y TIPOGRAFÍA (APA 7)
% -------------------------------------------------------------------------
% Márgenes: 2.54 cm (1 pulgada) en los 4 lados
\usepackage[margin=2.54cm]{geometry}

% Fuente: Times New Roman (para texto y matemáticas)
\usepackage{mathptmx}

% Espaciado: Doble espacio estricto
\usepackage{setspace}
\doublespacing

% Sangría: 1.27 cm (1/2 pulgada) en la primera línea
\setlength{\parindent}{1.27cm}

% -------------------------------------------------------------------------
% 3. ENCABEZADOS Y NUMERACIÓN
% -------------------------------------------------------------------------
\usepackage{fancyhdr}
\setlength{\headheight}{15.2pt} % Ajuste técnico para evitar advertencias
\pagestyle{fancy}
\fancyhf{} % Limpia estilos por defecto

% Encabezado Izquierdo: Título Corto (Running Head) - MÁX 50 CARACTERES
\fancyhead[L]{TÍTULO CORTO DE SU INVESTIGACIÓN}

% Encabezado Derecho: Número de página
\fancyhead[R]{\thepage}

% Sin línea horizontal decorativa (Norma APA)
\renewcommand{\headrulewidth}{0pt}

% -------------------------------------------------------------------------
% 4. JERARQUÍA DE TÍTULOS (5 NIVELES APA)
% -------------------------------------------------------------------------
\usepackage{titlesec}

% Nivel 1: Centrado, Negrita
\titleformat{\section}[block]{\centering\normalfont\normalsize\bfseries}{}{0em}{}
% Nivel 2: Alineado Izquierda, Negrita
\titleformat{\subsection}[block]{\raggedright\normalfont\normalsize\bfseries}{}{0em}{}
% Nivel 3: Alineado Izquierda, Negrita, Cursiva
\titleformat{\subsubsection}[block]{\raggedright\normalfont\normalsize\bfseries\itshape}{}{0em}{}
% Nivel 4: Sangría, Negrita, Punto final (Texto sigue en la línea)
\titleformat{\paragraph}[runin]{\normalfont\normalsize\bfseries}{}{0em}{}[.]
\titlespacing{\paragraph}{1.27cm}{0pt}{1em}
% Nivel 5: Sangría, Negrita, Cursiva, Punto final (Texto sigue en la línea)
\titleformat{\subparagraph}[runin]{\normalfont\normalsize\bfseries\itshape}{}{0em}{}[.]
\titlespacing{\subparagraph}{1.27cm}{0pt}{1em}

% -------------------------------------------------------------------------
% 5. TABLAS Y FIGURAS
% -------------------------------------------------------------------------
\usepackage{booktabs} % Para tablas profesionales (sin líneas verticales)
\usepackage{caption}  % Para personalizar los títulos (captions)

% Configuración de etiquetas (Tabla X / Figura X) y separadores
\DeclareCaptionLabelSeparator*{newline}{\\} % Salto de línea entre "Tabla 1" y el Título

% Configuración global para Figuras y Tablas
\captionsetup{
    labelsep=newline,        % Salto de línea
    justification=raggedright, % Alineado a la izquierda
    singlelinecheck=false,   % No centrar aunque sea corto
    labelfont=bf,            % Etiqueta en Negrita (Tabla 1)
    textfont=it,             % Título en Cursiva (Título de la tabla)
    skip=0.5em               % Espacio
}

% -------------------------------------------------------------------------
% 6. ÍNDICES (CONTENIDO, TABLAS, FIGURAS)
% -------------------------------------------------------------------------
\usepackage{tocloft}

% Eliminar numeración automática de secciones en el índice (1. Introducción -> Introducción)
\setcounter{secnumdepth}{0}

% Personalización del Índice General (TOC)
\renewcommand{\contentsname}{Tabla de contenido}
\renewcommand{\cfttoctitlefont}{\hfill\Large\bfseries} % Centrado
\renewcommand{\cftaftertoctitle}{\hfill}

% Personalización del Índice de Tablas (LOT)
\renewcommand{\listtablename}{Índice de tablas}
\renewcommand{\cftlottitlefont}{\hfill\Large\bfseries}
\renewcommand{\cftafterlottitle}{\hfill}
\renewcommand{\cfttabpresnum}{Tabla }     % Agrega "Tabla " antes del número
\renewcommand{\cfttabaftersnum}{.}        % Agrega punto final
\setlength{\cfttabnumwidth}{2.5cm}        % Espacio para la etiqueta

% Personalización del Índice de Figuras (LOF)
\renewcommand{\listfigurename}{Índice de figuras}
\renewcommand{\cftloftitlefont}{\hfill\Large\bfseries}
\renewcommand{\cftafterloftitle}{\hfill}
\renewcommand{\cftfigpresnum}{Figura }    % Agrega "Figura " antes del número
\renewcommand{\cftfigaftersnum}{.}
\setlength{\cftfignumwidth}{2.5cm}

% Líneas de puntos (...)
\renewcommand{\cftsecleader}{\cftdotfill{\cftdotsep}}

% -------------------------------------------------------------------------
% 7. PAQUETES ADICIONALES Y UTILITARIOS
% -------------------------------------------------------------------------
\usepackage{pdflscape}            % Para páginas horizontales
\usepackage{amsmath,amsfonts,amssymb} % Matemáticas avanzadas
\usepackage{graphicx}             % Imágenes
\usepackage{float}                % Para usar la posición [H]
\usepackage{csquotes}             % Comillas inteligentes
\usepackage{hyperref}             % Enlaces e hipervínculos
% Configuración de colores de enlaces (Negro para imprimir, funcional digitalmente)
\hypersetup{colorlinks=true,linkcolor=black,citecolor=black,urlcolor=black}

% -------------------------------------------------------------------------
% 8. GESTIÓN BIBLIOGRÁFICA
% -------------------------------------------------------------------------
\usepackage[natbibapa]{apacite} % Estilo APA nativo
\bibliographystyle{apacite}
\usepackage{bookmark}           % Mejora los marcadores del PDF

% =========================================================================
%                           INICIO DEL DOCUMENTO
% =========================================================================
\begin{document}

% --- 1. PORTADA ---
\pagenumbering{gobble}    % Apaga la numeración
\newgeometry{margin=3cm}  % Márgenes centrados para portada
% Asegúrese de tener el archivo Caratula.tex en la misma carpeta
% ==========================================
%           CARÁTULA INSTITUCIONAL
% ==========================================

\begingroup
    \centering
    \thispagestyle{empty} % Borra el número de esta página
    
    \vspace*{0.5cm}
    
    % --- ENCABEZADO UNMSM (NO MODIFICAR) ---
    {\bfseries\large UNIVERSIDAD NACIONAL MAYOR DE SAN MARCOS\par}
    {\scshape\small Universidad del Perú, Decana de América\par}
    \vspace{0.5cm}
    {\bfseries\large FACULTAD DE CIENCIAS ECONÓMICAS\par}
    \vspace{0.5cm}
    {\scshape\large Escuela Profesional de Economía Internacional\par}
    
    \vspace{1cm}
    
    % --- LOGO ---
    % Asegúrese de que el archivo 'unmsm insignia.png' esté en la carpeta del proyecto
    \includegraphics[width=4cm]{unmsm insignia.png} 
    
    \vspace{1cm}
    
    % --- TÍTULO DE LA INVESTIGACIÓN ---
    % Escriba el título en MAYÚSCULAS y respetando las tildes
    {\bfseries\LARGE [ESCRIBA AQUÍ EL TÍTULO DE SU TESIS O TRABAJO DE INVESTIGACIÓN]\par}

    \vspace{1.5cm}

    % --- ASESOR / PROFESOR (Opcional) ---
    {\large \textbf{Asesor:}}\par     
    \vspace{0.3cm}
    {\large [Grado y Nombre del Asesor] \par}
    
    \vspace{1cm}

    % --- AUTOR ---
    {\large \textbf{Presentado por:}}\par
    \vspace{0.3cm}
    {\large [Apellidos y Nombres del Estudiante] \par}
    
    \vfill
    
    % --- PIE DE PÁGINA ---
    {\large Lima, Perú\par}
    {\large 2026}
    
    \vspace*{1cm}
\endgroup
\newpage % Salto de página obligatorio para iniciar el contenido        
\restoregeometry          % Restaura márgenes APA
\clearpage

% --- 2. ÍNDICES PRELIMINARES ---
\newpage
\tableofcontents % Índice General

\newpage
\addcontentsline{toc}{section}{Índice de tablas}
\listoftables    % Índice de Tablas

\newpage
\addcontentsline{toc}{section}{Índice de figuras}
\listoffigures   % Índice de Figuras

% --- 3. RESUMEN Y ABSTRACT ---
\newpage
% La numeración y el encabezado se activan automáticamente aquí (Configuración fancyhdr)

% Resumen en Español
\section*{Resumen}
\addcontentsline{toc}{section}{Resumen}

\noindent % IMPORTANTE: Sin sangría al inicio del resumen
[Escriba aquí el resumen de su investigación en español. Recuerde que debe ser un solo párrafo, sin sangría, de entre 150 y 250 palabras, describiendo el objetivo, método, resultados y conclusiones principales.]

\vspace{0.5cm}

% Palabras clave con sangría
\indent \textit{\textbf{Palabras clave:}} palabra 1, palabra 2, palabra 3, palabra 4.

% Abstract en Inglés
\newpage
\section*{Abstract}
\addcontentsline{toc}{section}{Abstract}

\noindent % IMPORTANTE: Sin sangría
[Write here the abstract of your research in English. It must match the content of the Spanish abstract.]

\vspace{0.5cm}

\indent \textit{\textbf{Keywords:}} keyword 1, keyword 2, keyword 3, keyword 4.

\newpage

% =========================================================================
%                       CUERPO DEL DOCUMENTO
% =========================================================================
\pagenumbering{arabic} % Asegura numeración arábiga (1, 2, 3...)
\setcounter{page}{1}   % La Introducción comienza en la página 1

% --- CAPÍTULO 1 ---
\section{Introducción}
[Escriba aquí la introducción del trabajo...]

% --- CAPÍTULO 2 ---
\section{Revisión de Literatura}
[Escriba aquí el marco teórico y antecedentes...]

% --- CAPÍTULO 3 ---
\section{Metodología}
[Escriba aquí la descripción de datos, modelo econométrico y métodos...]

% --- CAPÍTULO 4 ---
\section{Resultados}
[Inserte aquí sus tablas y figuras con los resultados...]

% --- CAPÍTULO 5 ---
\section{Discusión y Conclusiones}
[Escriba aquí la discusión final...]

% =========================================================================
%                       REFERENCIAS BIBLIOGRÁFICAS
% =========================================================================
\newpage
% Asegúrese de tener su archivo 'bibliografia.bib' en la misma carpeta
\bibliography{bibliografia} 

% =========================================================================
%                       APÉNDICES
% =========================================================================
\newpage
\appendix % Cambia la numeración a letras (A, B, C...)
% Reinicia contadores de tablas/figuras por sección (Tabla A1, A2...)
\renewcommand{\thetable}{\thesection\arabic{table}}
\renewcommand{\thefigure}{\thesection\arabic{figure}}

% --- Apéndice A ---
\section{Instrumentos y Datos Adicionales}
\label{apendice:A}
\setcounter{table}{0} 
\setcounter{figure}{0}

[Incluya aquí material complementario, demostraciones matemáticas o tablas extra.]

\end{document}